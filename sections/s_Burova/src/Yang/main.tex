\documentclass{math-mech-sci}

\setdefaultlanguage{english}
\setotherlanguages{russian}

\deftitle%
  {Stochastic properties of a random Young diagram}

\defauthor%
  {Yang Yu.}% Yuqi
  {SPBU, Saint-Petersburg}%
  {st080797@student.spbu.ru}
\defauthor%
  {Yakubovich Y.}%
  {SPBU, Saint-Petersburg}%
  {y.yakubovich@spbu.ru}

\begin{document}

\maketitle

\begin{abstract}
This paper investigates the stochastic properties and dynamic evolution of random Young diagrams using advanced computer simulations, focusing on the application of the Corner\_w model with $w=1$. We explore how variations in initial parameters affect the growth and limit shapes of these diagrams. Our simulation methodology constructs Young diagrams incrementally, following probabilistic rules derived from theoretical models such as Rost’s limit shape theory.

We implement a novel color-coding technique to visually represent the frequency of individual boxes within the diagrams, using a gradient from dark green to white to indicate varying frequencies. This method clarifies the statistical distribution of elements within the diagrams and enhances our understanding of their probabilistic behaviors over multiple iterations. Our findings show that as the number of iterations increases, the emergent Young diagrams exhibit a stable limit shape consistent with theoretical predictions.

Additionally, our simulations reveal the aggregate behavior of multiple Young diagrams, providing new insights into their collective dynamics and structural properties when observed as an ensemble. The higher frequency of box appearances is indicated by lighter shades, moving towards white for boxes appearing in every iteration, thus offering an intuitive and interactive way to analyze these complex structures.

In the theoretical exploration of jeu de taquin, we developed new combinatorial insights by examining the interaction dynamics of dual paths within a Young tableau. Our findings reveal specific forbidden movement patterns and interaction scenarios between these paths, enhancing the understanding of their complex behavior. Key results include proofs of several lemmas detailing these forbidden configurations and their implications for the transformation dynamics of jeu de taquin paths.

Additionally, we propose a conjecture on the stability of endpoint distributions under the jeu de taquin transformation, supported by empirical data from the simulations. This conjecture suggests predictable patterns in the development of jeu de taquin paths, potentially contributing significantly to the theoretical framework of combinatorial dynamics in Young tableaux.

These contributions extend the current understanding of Young diagrams and jeu de taquin dynamics, opening new avenues for research in probabilistic and combinatorial mathematics.
\end{abstract}
\begin{thebibliography}{8}
\bibitem{RomikSniady2015} Dan Romik and Piotr Sniady, JEU DE TAQUIN DYNAMICS ON ININITE YOUNG TABLEAUX AND SECOND CLASS PARTICLES, The Annals of Probability, 2015, Volume 43, Pages 682-737, Number 2.

\bibitem{ErikssonSjostrand2012} Eriksson, K. and Sj\"ostrand, J., Limiting shapes of birth-and-death processes on {Y}oung diagrams, Advances in Applied Mathematics, Volume 48, Pages 575-602, Year 2012.

\bibitem{Vershik1996} A. M. Vershik, Statistical mechanics of combinatorial partitions, and their limit shapes, Funct. Anal. Appl., 1996, Volume 30, Pages 90--105.

\bibitem{Simon1955} H. Simon, On a class of skew distribution functions, Biometrika, 1955, Volume 42, Pages 425--440.

\bibitem{Rost1981} H. Rost, Non-equilibrium behaviour of a many particle process: Density profile and local equilibria, Probab. Theory Related Fields, 1981, Volume 58, Pages 41--53.

\bibitem{Ewens2004} W.J. Ewens, Mathematical Population Genetics, Springer, 2004, Edition second.

\bibitem{AmirAngelValko2011} Gideon Amir and Omer Angel and Benedek Valk\'o, The TASEP Speed Process, The Annals of Probability, 2011, Volume 39, Number 4, Pages 1205--1242, DOI: 10.1214/10-AOP561.

\bibitem{MountfordGuiol2005} Mountford, T. and Guiol, H., The motion of a second class particle for the TASEP starting from a decreasing shock profile, Ann. Appl. Probab., 2005, Volume 15, Pages 1227--1259.

\end{thebibliography}

\end{document}
