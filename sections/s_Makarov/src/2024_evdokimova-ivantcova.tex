% !Mode:: "TeX:UTF-8"
% -*-coding: utf-8 -*-

% Русский язык дополнительно настраивать не нужно.
% Убедитесь, что ваш редактор поддерживает UTF-8.
\documentclass{math-mech-sci}

% Название и авторы задаются при помощи специальных
% команд \deftitle и \defauthor. Специальные команды
% требуются для генерации корректной метаинформации в
% PDF-файле.

% Блоки в квадратных скобках опциональны, они предназначены
% для сносок на гранты, которыми поддержана работа.

\deftitle%
   {О курсе <<Методика преподавания компьютерных наук>>
в непедагогических вузах}

\defauthor%
  {Евдокимова Т.О.}%
  {СПбГУ, Санкт-Петербург}%
  {t.evdokimova@spbu.ru}
\defauthor%
  {Иванцова О.Н.}%
  {СПбГУ, Санкт-Петербург}%
  {o.ivancova@spbu.ru}

\begin{document}

\maketitle

\begin{abstract}
На математико-механическом факультете СПбГУ курсы по методике преподавания читаются на большинстве образовательных программ. Это позволяет студентам овладеть не только своей <<рабочей>> специальностью, но и, в широком смысле, осознать способы её освоения, в том числе, начальные шаги, которые могут быть осуществлены и в школьном обучении. В связи с этим, закономерен интерес к наличию и реализации аналогичных курсов в других профильных, но не педагогических вузах РФ. В данной работе представлено небольшое исследование, проведенное авторами по этому вопросу.
\end{abstract}

\section{История вопроса}

В СПбГУ с момента открытия образовательной программы <<Математическое обеспечение и администрирование информационных систем>> для магистров читается обязательный курс <<Методика преподавания компьютерных наук>>. 

Ранее, с 2010 года, для студентов 5-го курса специалитета проводился элективный курс <<Методика преподавания информатики и ИКТ>>.
Освоение данной дисциплины дает возможность овладеть навыками преподавания, в том числе и для школьников.

Идея разработки и реализации подобного курса была заложена ещё коллегами-математиками, полагающими необходимость профильной подготовки учителей для специализированных школ, которую затруднительно дать в педагогическом вузе.

В связи с этим у авторов возник вопрос о том, реализуются ли подобные курсы в других крупных (ведущих) непедагогических вузах?

\section{Результаты исследования}

%\subsection{}

Были рассмотрены учебные планы следующих вузов:
\begin{itemize}
 \item Санкт-Петербургский государственный университет
 \item Московский государственный университет 
 \item Новосибирский государственный университет 
 \item Национальный исследовательский университет ИТМО 
 \item Санкт-Петербургский государственный электротехнический университет <<ЛЭТИ>> 
% \item Московский физико-технический институт (МФТИ)
\end{itemize}
При этом рассматривались образовательные программы, связанные с вычислительной техникой, программированием и математикой.

Опираясь на данные открытого доступа, удалось выяснить, что в рассматриваемых вузах отсутствуют курсы по методике преподавания компьютерных наук.
Анализ проводился по основным образовательным программам как бакалавриата, так и магистратуры.


\subsection{Другие результаты}

Был проведен анализ реализации курсов по методике преподавания компьютерных наук в других непедагогических вузах.
В результате получен список следующих вузов:

\begin{itemize}
\item Кубанский государственный университет (КубГУ)
\item Рязанский государственный университет (РГУ)
\item Саратовский государственный университет (СГУ)
\end{itemize}
Соответствующие дисциплин этих вузов --- это <<Методика преподавания ИКТ>>, <<Методика преподавания компьютерных наук в высшей школе>> и <<Методика обучения информатике>>. 

% 
% Другие характеристики приведены в таблице\ref{tempo}. Столбец <<О/Э>> означает обязательность или элективность курса.
%  
% \begin{table}[h]
% \begin{center}
% \begin{tabular}{|l|%p{5cm}
% c|c|c|c|c}}
% \hline
% Вуз    & Направление  & О/Э & зач/экз & семестр & часы  \\
% \hline
% СПбГУ & Математическое обеспечение и администрирование информационных систем & О & зачет & 3 & 30 \\
% \hline
% КубГУ& Прикладная математика и информатика & О & экзамен & 1 & 42 \\
% \hline
% РГУ& Прикладная математика и информатика & Э & зачет & 3 & 54 \\
% \hline
% СГУ& Прикладная математика и информатика & Э & экзамен & 2 & 48 \\
% \hline
% \end{tabular}
% \caption{Формы отчетности и количество часов дисциплин}\label{tempo}
% \end{center}
% \end{table}



% \begin{table}[h]
% \begin{center}
% \begin{tabular}{|r|l|}
% \hline
% \thd{Показатель} & \thd{Значение} \tabularnewline
% \hline
% Всего секционных докладов & 130 \tabularnewline
% \hline
% Всего секций & 17 \tabularnewline
% \hline
% Страниц в сборнике & 480 \tabularnewline
% \hline
% \end{tabular}
% \caption{Количественные показатели конференции
%   СПИСОК-2011}\label{tab:math-science2011}
% \end{center}
% \end{table}

\section{Характеристика курсов}

{\it Кубанский государственный университет (КубГУ)} реализует дисциплину «Методика преподавания ИКТ» для магистров по направлению подготовки 01.04.02 «Прикладная математика и информатика» по трем образовательным программам:
\begin{enumerate}
\item «Математические и информационные технологии в цифровой экономике»
\item «Математическое моделирование в естествознании и технологиях»
\item «Технологии программирования и разработки информационно-коммуникационных систем»
\end{enumerate}

Все образовательные программы предназначены для магистров Факультета компьютерных технологий и прикладной математики трех кафедр:
\begin{itemize}
\item Кафедра Прикладной математики
\item Кафедра Математического моделирования 
\item Кафедра Информационных технологий
\end{itemize}

Рассматриваемая дисциплина является обязательной и читается в первом семестре, продолжительность –-- 42 ауд. часа. По завершении проводится экзамен.
Из аннотации к методическому пособию по данной дисциплине следует, что основное внимание уделяется психолого-педагогическим проблемам преподавателей ИТ; конкретным методикам преподавания программных сред, включая рекомендации по проведению лекционных и семинарских занятий, по конструированию практических заданий и нормативно-правовым вопросам преподавательской деятельности в высшей школе.
К сожалению, при более тщательном поиске аннотации и РПД в открытом доступе не найдено!

{\it Рязанский государственный университет (РГУ)} реализует дисциплину «Методика преподавания компьютерных наук в высшей школе» для магистров по направлению подготовки 02.04.02 «Фундаментальная информатика и информационные технологии» по образовательной программе «Информационные системы» на Кафедре информатики, вычислительной техники и методики преподавания информатики Института физико-математических и компьютерных наук.

Рассматриваемая дисциплина является элективной и читается в третьем семестре, продолжительность –-- 54 ауд. часа (лекции – 18, лабораторные работы --– 36). По завершении проводится зачет.

Проанализировав РПД можно сказать, что в данной дисциплине основное внимание уделяется методологическим основам методики преподавания компьютерных наук; организационным формам обучения в вузе; контролю и оценке знаний студентов.

В ходе изучения рассматриваемой дисциплины делается акцент на основные задачи, решаемые российской высшей школой при переходе на двухуровневую систему образования, на общее понятие, задачи и функции методики преподавания в высшей школе. Уделяется внимание историческому развитию и становлению современной методической системы, а также взаимосвязи образовательной, воспитательной и развивающей функций обучения. Рассматриваются психологические основы учебного процесса: мотивы учения студентов, их развитие и формирование; единство преподавания и учения; обучение как сотворчество преподавателя и студентов. Магистрами изучаются современные тенденции развития образования; методологические основы и организация педагогического процесса и инновации в образовании.
Подробно рассматриваются такие понятия, как «метод», «прием», «средство» обучения и «педагогическая технология», а также классификация методов обучения. Изучается взаимосвязь методов обучения и условия их оптимального выбора, а также виды педагогических технологий (технологии традиционного обучения, компьютерные технологии, технологии модульного и контекстного обучения, интенсивная технология обучения).

Уделено внимание различным видам контроля в вузе (оперативный, текущий, рубежный, итоговый) и формам проведения: зачеты, экзамены, коллоквиумы, Интернет-экзамены, тестирование, контрольные работы, защиты рефератов, курсовых и дипломных работ, а также особенностям рейтингового контроля и оценки достижений студентов, с учетом его достоинств и недочетов. Рассматриваются и нетрадиционные формы и методы контроля в образовательном процессе.

Лабораторные работы посвящены разработкам модели академического занятия; эффективности методов обучения; использованию средств медиа в обучении; диагностике степени обученности студентов и сформированности профессиональной мотивации студентов.


{\it Саратовский государственный университет (СГУ)} реализует дисциплину «Методика преподавания компьютерных наук» для магистров по направлению подготовки 09.04.01 «Информатика и вычислительная техника» по двум образовательным программам: «Сети ЭВМ и телекоммуникации» и «Анализ и синтез распределенных технических систем» на Факультете компьютерных наук и информационных технологий.
Рассматриваемая дисциплина является обязательной и читается во втором семестре, продолжительность –-- 36 ауд. часа (лекции --– 28, лабораторные работы –-- 8 часов). По завершении проводится экзамен.

Проанализировав РПД можно сказать, что в данной дисциплине основное внимание уделяется IT-образованию; различным аспектам педагогике и дидактике высшей школы; разнообразным методикам.
Подробно рассматриваются такие понятия, как «образование», «образовательное пространство» и «образовательная среда». Изучается специфика IT-образования и мировые тенденции его развития. 
Особое внимание уделяется понятию «педагогика высшей школы». Рассматривается предмет, задачи и методология педагогики высшей школы, а также структура, особенности, закономерности и принципы педагогического процесса в вузе. Изучается специфика педагогического процесса в условиях электронного и дистанционного обучения; включение бизнес-структур в педагогический процесс подготовки IT-специалистов. 
Говорится об основных документах, регламентирующих педагогический процесс и деятельность преподавателей вузов, о ФГОС ВО, о корпоративных и профессиональных стандартах, о положениях, регламентирующих учебный процесс в вузе. Рассмотрены рабочие учебные планы и рабочие программы дисциплин; тематические планы и учебно-методические комплексы.

Уделено внимание и понятию «дидактика высшей школы». Рассматривается структура обучения студента вуза; теории, концепции и технологии обучения в высшей школе, а также инновационные технологии обучения. Изучается специфика обучения компьютерным наукам студентов разных направлений, возрастов и другие вопросы индивидуализации обучения. 

Методическая система обучения студентов компьютерным наукам предполагает изучить методы обучения, виды учебных занятий, средства обучения, специфику средств обучения компьютерным наукам, а также роль лекции, семинара и практических занятий в учебном процессе вуза, методику работы над ними.
Даются рекомендации по формированию учебно-методических комплексов (УМК) учебных дисциплин и разработке частных методик для дисциплин компьютерного цикла в соответствии с требованиями ФГОС ВО.
Лабораторные работы посвящены созданию собственного электронного курса на основе изученного материала.





{\it В Санкт-Петербургском государственном университете (СПбГУ)} курс методики преподавания компьютерных наук является обязательным для магистров второго курса направления <<Математическое обеспечение и администрирование информационных систем>>, состоит из 16 часов лекций и 16 часов практики и завершается зачетом. Курс посвящен знакомству студентов как с методикой преподавания информатики в школе, так и с различными дидактическими подходами, применимыми к большинству дисциплин. Перечислим основные источники, обсуждаемые в этом курсе.

Базовый школьный курс информатики, его цели, задачи, содержание и методика преподавания рассматриваются по учебниками Кушниренко~А.\,Г., Леонова~А.\,Г., Зайдельмана~Я.\,Н., Тарасовой~В.\,В. и по методическим материалам, изложенным в книге Кушниренко~А.\,Г. и Лебедева~Г.\,В. <<12 лекций о том, для чего нужен школьный курс информатики и как его преподавать>>. Дополнительным источником задач и теории к ним является книга Кушниренко~А.\,Г. и Лебедева~Г.\,В. <<Программирование для математиков>>. Эти материалы выгодно отличаются обоснованием построения курса, рассматриваемых авторами тем и задач к ним.

Занятия по информатике в начальной школе изучаются по работам Первина~Ю.\,А., в частности, по его книге <<Методика раннего обучения информатике>>, в которой автор тоже обосновывает свой выбор тем, выбор целей и задач курса. Данный материал может быть дополнен разбором занятий с ресурса урокцифры.рф.

В качестве учебника по углубленному курсу информатики в старшей школе изучается комплект учебников Полякова~К.\,Ю. и Еремина~Е.\,А., содержащий разнообразный набор тем и задач различной сложности.

Основной обсуждаемый дидактический подход --- это Большая Дидактика Е.\,Яновицкой и М.\,Адамского. Этот подход интересен и полезен тем, что его разработали и применяют в различных школах, в том числе и неспециализированных; он может быть адаптирован на большинство школьных предметов (и часть вузовских); его идеи и методы позволяют качественно обучать всех и особенно полезны и эффективны для начала обучения предмету и в случаях слабой подготовки учащихся.

Другой методической книгой является пособие Гина А.\,А. <<Приемы педагогической техники: свобода выбора, открытость, деятельность, обратная связь, идеальность>>, выдержавшее уже 18 переизданий. Книга содержит наборы дидактических приемов и приемов управления классом, которые можно комбинировать друг с другом, тем самым разнообразя форму проведения уроков разных типов.

Методическую копилку дополняет книга Ершова~П.\,М., Ершовой~А.\,П. и Букатова~В.\,М. <<Общение на уроке, или
Режиссура поведения учителя>>, в которой систематизированы виды межличностного взаимодействия и приводятся примеры соответствующих педагогических ситуаций. Дальнейшее овладение этой тематикой возможно на основе других работ её авторов.

Дополнительными источниками, которые можно рекомендовать слушателям данного курса, являются <<Школа будущего, построенная вместе с детьми>> А.\,Н.\,Тубельского, <<Школа влияния>> С.\,Соловейчика, <<Другая школа: образование --- не система, а люди>> А.\,И.\,Мурашева, а также глава 14 <<Об учении, преподавании и обучении преподаванию>> известной книги Д.\,Пойа <<Математическое открытие>>.


\section{Заключение}

В работе представлены краткие характеристики курсов методики преподавания информатики/компьютерных наук, читаемых в некоторых непедагогических вузах для обучающихся по IT-специальностям.


\begin{thebibliography}{9}

\bibitem{spbu} Учебные планы и рабочие программы дисциплин СПбГУ: 

  \href{https://spbu.ru/sveden/education}{https://spbu.ru/sveden/education}

\bibitem{mgu} Учебные планы и рабочие программы дисциплин МГУ: 

  \href{http://edu.msu.ru/depts.shtml}{http://edu.msu.ru/depts.shtml}
  
\bibitem{ngu} Учебные планы и рабочие программы дисциплин НГУ: 

 \href{https://www.nsu.ru/n/information-technologies-department/education_fit/programs/}{https://www.nsu.ru/n/information-technologies-department/education\_fit/programs/}

\bibitem{itmo} Учебные планы и рабочие программы дисциплин ИТМО: 

 \href{https://abit.itmo.ru/master}{https://abit.itmo.ru/master}

\bibitem{leti} Учебные планы и рабочие программы дисциплин ЛЭТИ: 

  \href{https://etu.ru/sveden/education/eduop/}{https://etu.ru/sveden/education/eduop/}

\bibitem{nngu} Учебные планы и рабочие программы дисциплин ННГУ: 

  \href{http://www.unn.ru/sveden/education/edu-op.php}{http://www.unn.ru/sveden/education/edu-op.php}
  
%---------------

\bibitem{kubgu} Учебные планы и рабочие программы дисциплин КубГУ: 

  \href{https://www.kubsu.ru/ru/education/programs}{https://www.kubsu.ru/ru/education/programs}
  
  \bibitem{rgu} Учебные планы и рабочие программы дисциплин РГУ: 

  \href{https://www.rsu.edu.ru/sveden/education/}{https://www.rsu.edu.ru/sveden/education/}
  
  \bibitem{sgu} Учебные планы и рабочие программы дисциплин СГУ: 

  \href{https://www.sgu.ru/education/courses}{https://www.sgu.ru/education/courses}
  
  
 \end{thebibliography}

\end{document}
